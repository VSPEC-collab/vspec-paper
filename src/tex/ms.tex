% Define document class
\documentclass[twocolumn]{aastex631}
\usepackage{showyourwork}


\newcommand{\teff}{$T_{\rm eff}$}
\definecolor{tedcommentcolor}{HTML}{e17701}
\newcommand{\TJ}[1]{\textcolor{tedcommentcolor}{#1}}

\shorttitle{\sc VSPEC}
\shortauthors{Johnson et al.}

% Begin!
\begin{document}

% Title
\title{The {\sc VSPEC} code: Spectroscopic phase curves of 3D exoplanet models in the presence of stellar variability}

\author{Ted Johnson}
\affiliation{NASA Goddard Space Flight Center \\
8800 Greenbelt Rd \\
Greenbelt, MD 20771, USA}
\affiliation{Nevada Center for Astrophysics, University of Nevada, Las Vegas, 4505 South Maryland Parkway, Las Vegas, NV 89154, USA}
\affiliation{Department of Physics and Astronomy, University of Nevada, Las Vegas, 4505 South Maryland Parkway, Las Vegas, NV 89154, USA}
% Also CRESST, SURA, SEEC!

\author{Cameron Kelahan}
\affiliation{NASA Goddard Space Flight Center \\
8800 Greenbelt Rd \\
Greenbelt, MD 20771, USA}

\author{Avi M. Mandell}
\affiliation{NASA Goddard Space Flight Center \\
8800 Greenbelt Rd \\
Greenbelt, MD 20771, USA}

\author{Tom Barclay}
\affiliation{NASA Goddard Space Flight Center \\
8800 Greenbelt Rd \\
Greenbelt, MD 20771, USA}

\author{Veselin B. Kostov}
\affiliation{NASA Goddard Space Flight Center \\
8800 Greenbelt Rd \\
Greenbelt, MD 20771, USA}

\author{Geronimo L. Villanueva}
\affiliation{NASA Goddard Space Flight Center \\
8800 Greenbelt Rd \\
Greenbelt, MD 20771, USA}

% Abstract with filler text
\begin{abstract}
    We present the Variable Star Phase Curve (\href{https://github.com/VSPEC-collab/VSPEC}{ \sc VSPEC}) code,
    a python package to simulate spectroscopic observations of rocky exoplanets in the presence of stellar variability.
    {\sc VSPEC} uses the Planetary Spectrum Generator's Global Emission Spectra (PSG GlobES) application along with a custom-built
    stellar model to produce spectroscopic light curves of the planet-host system. We also describe associated codes. \TJ{These codes should be named here and described a little}
\end{abstract}

\keywords{Exoplanet, M dwarf}

% Main body with filler text
\section{Introduction}
\label{sec:intro}

In the era of high-sensitivity exoplanet characterization missions such as the James Webb Space Telescope (JWST)
and the Atmospheric Remote-sensing Infrared Exoplanet Large-survey (ARIEL), spectral analysis of exoplanet atmospheres
is increasingly sensitive to contamination due to stellar inhomogenies (spots, granulation, etc.).

Recent analysis of the JWST/NIRSpec transit of GJ 486b by \citet{moran2023} exposed a degeneracy between
atmospheric absorption by water and water-rich spots on the stellar surface. This ``transit light source effect''
\citep[TLS,][see also \citet{apai2018,barclay2021,garcia2022,barclay2023}]{rackham2018}
occurs when the occulted region of the stellar surface is not representitive of the disk-integrated spectrum

Similarly, the Mid-IR Exoplanet CLimate Explorer \citep[MIRECLE,][]{mandell2022} mission concept
will employ the Planetary Infrared Excess \citep[PIE][]{stevenson2020} technique to extract the planetary
contribution from combined-light observations. Uncertainties on the stellar spectrum will dominate the analysis if it is not removed appropriately.

To adequately prepare for future observations and future missions it will be necessary to demonstrate a method to mitigate these effects.
This task requires a flexible tool that combines models of exoplanet atmospheres and stellar variability in a robust way.

We present {\sc VSPEC}: Variable Star PhasE Curve\footnote{\url{https://github.com/VSPEC-collab/VSPEC}},
an open-source Python 3 package to simulate observations of exoplanet systems with variable host stars.
{\sc VSPEC} combines NASA's Planetary Spectrum Generator \citep[PSG,][]{villanueva2018} with a grid of
custom PHOENIX models \citep{husser2013} to produce multiwavelength phase curves of the system, accounting
for a 3D planetary atmosphere, transits and eclipses, and multiple sources of stellar variability.

In this paper we will describe the science behind {\sc VSPEC} and demonstrate examples of its use.
Section \ref{sec:star} describes the stellar model and the available sources of variability.
Section \ref{sec:pl_model} discusses modeling the planetary atmosphere, including PSG/GlobES and the
built-in GCM. In Section \ref{sec:examples} we provide examples of {\sc VSPEC} use cases and demonstrate
its value to the exoplanet community. Finally, in Section \ref{sec:conclusion} we will discuss the future of the code and issues it might have.


{Stellar Model \label{sec:star}}

The {\sc VSPEC} star model is designed modularly to allow
for both simple and complex behaviors. Currently, it
is represented by a rectangular grid of points on the stellar surface,
each assigned an effective temperature. At any given time, the model computes
the surface coverage fractions of each temperature visible to the observer, accounting for the
spherical geometry, limb darkening, and any occultation by a transiting planet.

The attributes of the \texttt{Star} class describe the bulk properties of the star, including radius,
period, and the effective temperature of quiet photosphere. Herein we refer to this temperature as the
photosphere temperature to differentiate it from the temperature of spots, faculae, or other sources of variability.

\begin{figure}
    \centering
        \includegraphics[width=0.5\textwidth]{figures/surface_map_and_lc.pdf}
    \script{surface_map_and_lc.py}
    \caption{{\bf Top}: The surface map of a star with spots and faculae during a transit.
    The large circles are spots. Faculae display their 3D structure, as the hot wall is more prevalent near the limbs.
    {\bf Bottom}: An example lightcurve of a spotted star.}
    \label{fig:surface_map}
\end{figure}


\bibliography{syw,VSPEC}

\end{document}
